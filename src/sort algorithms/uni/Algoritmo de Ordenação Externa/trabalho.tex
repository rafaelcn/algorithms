% Created 2018-06-19 Ter 07:46
\documentclass[a4paper, 12pt]{report}
\usepackage[utf8]{inputenc}
\usepackage[T1]{fontenc}
\usepackage{fixltx2e}
\usepackage{graphicx}
\usepackage{longtable}
\usepackage{float}
\usepackage{wrapfig}
\usepackage{rotating}
\usepackage[normalem]{ulem}
\usepackage{amsmath}
\usepackage{textcomp}
\usepackage{marvosym}
\usepackage{wasysym}
\usepackage{amssymb}
\usepackage{hyperref}
\tolerance=1000
\usepackage{minted}
\usepackage{indentfirst}
\usepackage[portuguese, english]{babel}
\usepackage{setspace}
\usepackage{fancyhdr}
\usepackage{float}
\usepackage{url}
\usepackage[utf8]{inputenc}
\usepackage{minted}
\renewcommand\listingscaption{Código}
\author{Rafael Campos Nunes  \\ Mikael Messias}
\date{12 de junho de 2018}
\title{Análise sobre Algoritmos de Ordenação Externa}
\hypersetup{
  pdfkeywords={},
  pdfsubject={},
  pdfcreator={Emacs 25.3.1 (Org mode 8.2.10)}}
\begin{document}

\maketitle
\#+LANGUAGE: pt-br

\part{Introdução}
\label{sec-1}

Atualmente algoritmos são utilizados em diversas atividades corriqueiras,
mesmo que não conscientemente observadas são realizadas atividades como a
rota para um local específico, buscar algum número de telefone no \emph{smartphone}
ou a busca por um amigo em alguma rede social. Todos esses algoritmos
utilizam-se de algum método de ordenação e/ou busca, importante ressaltar ainda
que estes ordenam sobre dados que por vezes não estão presentes em memória
primária do dispositivo e sim localizados em memória de armazenamento em massa.

O presente trabalho realiza a análise dos algoritmos de ordenação em memória
secundária, denominados na literatura de: algoritmos de ordenação externa.
Esses algoritmos operam sobre dados não presentes na memória primária, isto
porque na maioria das vezes os dados sobre quais esses algoritmos operam são
muito grandes para a capacidade da memória primária e algumas estratégias são
utilizadas para essa categoria de ordenação. Apesar dos referidos algoritmos
trabalharem em memória secundária os dados são ordenados em memória primária,
contudo, em menor quantidade.

\part{Análise do Merge Sort}
\label{sec-2}
\chapter{Implementação do algoritmo}
\label{sec-2-1}

O \emph{Merge} \emph{Sort} externo é um algoritmo que rearranja os registros de um
arquivo da memória secundária do computador utilizando uma estratégia de
ordenação interna. Faz a leitura dos registros no arquivo especificado a ser
ordenado e possui como principais etapas a classificação e a intercalação.
Ao fim da execução, o algoritmo gera um arquivo de saída com os registros
ordenados na memória secundária do computador.

\section{Etapa de classificação}
\label{sec-2-1-1}

A ordenação externa torna-se necessária quando se tem um arquivo cujo número
de registros é maior do que a memória primária pode armazenar. Esses registros
precisam ser classificados em blocos menores, de forma que possam ser
rearranjados utilizando alguma estratégia de ordenação interna.

Na implementação do algoritmo, para este trabalho, a memória primária é
representada como um vetor de números inteiros, com tamanho máximo igual a 100.
Portanto, independente da quantidade registros do arquivo original, o \emph{Merge}
\emph{Sort} externo só será capaz de criar blocos de ordenação com 100 elementos.

Após a cópia dos registros para a memória interna, um algoritmo de ordenação
interna é aplicado ao vetor que, logo em seguida, tem seus valores salvos
em um arquivo temporário, tal arquivo é denominado como um bloco ordenado. O
algoritmo \emph{Heap} \emph{Sort} foi utilizado para ordenação dos blocos na memória
primária.

\section{Etapa de intercalação}
\label{sec-2-1-2}

Após classificar os registros do arquivo original em blocos ordenados, é
necessário combiná-los em um único bloco ordenado. Esse etapa é chamada
intercalação.

No algoritmo, há uma estrutura cuja função é auxiliar a intercalação
armazenando algumas informações relevantes durante o processo.

\begin{minted}[frame=lines,linenos=true]{c}
struct file {
    FILE *f;
    int pos;
    int MAX;
    int *buffer;
};
\end{minted}

O primeiro passo é criar um vetor da estrutura \texttt{File} com tamanho igual ao
número de blocos gerados pela etapa de classificação. O campo buffer,
que referencia um valor ou conjunto de valores inteiros, será utilizado para
armazenar os registros do arquivo referenciado por \texttt{f}.

\begin{enumerate}
\item Selecione o menor registro na posição atual de cada bloco.
\item Armazene o valor dentro do vetor na memória primária.
\item Quando o vetor atingir seu tamanho máximo, salve os registros dentro do
arquivo de saída,
\item Repita os passos até que não haja mais valores para serem intercalados.
\end{enumerate}

\chapter{Manipulação de arquivos}
\label{sec-2-2}

O \emph{Merge} \emph{Sort} externo requer leitura e escrita em arquivos na memória secundária
do computador em alguns momentos da execução.  O algoritmo implementado possui
funções específicas para manipulação de arquivos, que permite realizar essas
operações.

A função de inserção anexa ao fim de um arquivo os valores de um vetor de
números inteiros. É utilizada tanto na classificação, quando são gerados
blocos menores que serão posteriormente ordenados e escritos em arquivos
temporários, quanto na intercalação, quando os blocos serão combinados
novamente, gerando o arquivo de saída.

Durante a etapa de intercalação, como os blocos ordenados estão
armazenados em arquivos na memória secundária do computador, é necessário
utilizar vetores para recuperar os registros de cada bloco afim de
intercalá-los e escrevê-los no arquivo de saída. A função de recuperação,
a partir de um arquivo localizada na memória secundária  do computador,
preenche um vetor de números inteiros com os registros desse arquivo.

\chapter{Análise do algoritmo}
\label{sec-2-3}

\part{Merge Sort utilizando Fila de Prioridades}
\label{sec-3}

A fila de prioridades é uma estrutura de dados em que os elementos desta
obedecem uma propriedade, sendo esta a característica que confere à fila de
prioridades, conhecida como heap, a denominação de \emph{heap de máximo} ou \emph{heap
de mínimo}. A propriedade enunciada é similar nos dois casos de \emph{heap} com a
troca somente da condição da propriedade.

\begin{enumerate}
\item Todo \emph{i-ésimo} elemento é maior que o elemento \emph{2*i-ésimo+1}
\item Este \emph{i-ésimo} elemento é também maior que o elemento \emph{2*i-ésimo+2}
\end{enumerate}

Tal estrutura é utilizada para na etapa de classificação e intercalação de
elementos. Sua utilização traz alguns benefícios pois não necessariamente os
blocos de arquivo serão divididos igualmente, por vezes sendo utilizados menos
arquivos.

A implementação da fila de prioridades se concretizou por meio de duas
estruturas, elas são mostradas no código abaixo.

\begin{listing}[H]
\begin{minted}[frame=lines,linenos=true]{c}
struct heap_element {
    int key;
    int weight;
};

struct priority_queue {
    // The vector that holds the heap inside the structure
    struct heap_element *vector;

    // The current index of the heap
    uint16_t index;
    // The current size of the heap
    uint16_t size;
};

typedef struct priority_queue Heap;
typedef struct heap_element HeapElement;
\end{minted}
\caption{Representação da fila de prioridades em C}
\end{listing}

A primeira estrutura representa um elemento da fila de prioridades e a segunda
representa a fila em si. As diversas operações realizadas sobre essa estrutura
estão todas descritas no arquivo \emph{heap.c}. A saber, as operações realizam
as funções de inserção, inserção em um intervalo especificado, remoção do
primeiro elemento e a ordenação dos elementos desta fila.

Tais operações são fundamentais nas etapas de classificação e intercalação de
dados dado que a ordenação de seleção por meio de substituição utiliza a
referida estrutura. Este algoritmo utiliza os seguintes passos para fazer
a leitura e ordenação dos arquivos nas diversas fitas de dados:

\begin{enumerate}
\item Lê os números de um arquivo, preenchendo o espaço permitido na memória primária
\item Retira-se o primeiro elemento da fila, aqui chamado de $v_0$ para a leitura do próximo
\item Compara-se o $v_0$ com o elemento a ser inserido, se este for menor aumenta-se o peso
\item Insere-se o último número lido na primeira posição do $heap$
\item Escreve-se o elemento $v_0$ na fita correspondente ao seu peso
\end{enumerate}

\chapter{Operações na Fila de Prioridades}
\label{sec-3-1}

Como enunciado anteriormente as operações na fila são quatro, nesta seção elas
são trabalhadas com mais detalhes. A primeira delas é a inserção em um
intervalo específico da fila, a configuração do algoritmo se dá como a seguir:

\begin{listing}[H]
\begin{minted}[frame=lines,linenos=true]{c}
void sift_up_i(Heap *heap_t, int key, int weight, int index) {

    if (heap_t == NULL) {
        return;
    }

    HeapElement he;

    he.key = key;
    he.weight = weight;

    heap_t->vector[index] = he;

    heap_sort(heap_t);
}
\end{minted}
\caption{Inserção de elementos com pesos na fila de prioridades}
\end{listing}

O código faz uma verificação de segurança e após isso insere o valor criado
na posição especificada pelo chamador da função. A operação posterior
utiliza-se desta função para a inserção de novos elementos, dessa forma
aumentando a coesão do código e reaproveitamento deste.

\begin{listing}[H]
\begin{minted}[frame=lines,linenos=true]{c}
void sift_up(Heap *heap_t, int key, int weight)
    if (heap_t == NULL) {
        return;
    }

    sift_up_i(heap_t, key, weight, heap_t->index);

    heap_t->index = heap_t->index+1;
    heap_t->size = heap_t->index;
}
\end{minted}
\caption{Inserção de elementos com pesos na fila de prioridades}
\end{listing}

A remoção de um elemento na fila é realizada de maneira simples, só é
necessário recuperar o primeiro dado dessa fila e retorná-lo. Observa-se,
entretanto, que é necessário, logo após a remoção deste elemento a inserção
de outro no mesmo lugar. A API da fila foi projetada dessa maneira para
melhor atender às necessidades algoritmicas dos projetistas.

\begin{listing}[H]
\begin{minted}[frame=lines,linenos=true]{c}
HeapElement sift_down(Heap* heap_t) {

    HeapElement e = heap_t->vector[0];

    heap_sort(heap_t);

    return e;
}
\end{minted}
\caption{Remoção de elementos na fila de prioridades}
\end{listing}

\chapter{Ordenação da Fila de Prioridades}
\label{sec-3-2}

A fila de prioridades, além de ter todas as operações já denotadas, contém
um método de ordenação diferente pois necessita ordenar os elementos a partir
de seus pesos e quando não se delinea diferença entre estes é utilizado o
valor das chaves. O algoritmo criado para tal fim é mostrado abaixo:

\begin{listing}[H]
\begin{minted}[frame=lines,linenos=true]{c}
void heapify(HeapElement *heap_elements, int n, int i) {

    int smallest = i;

    int l = 2*i+1;
    int r = 2*i+2;

    int smallest_weight = heap_elements[smallest].weight;

    if (l < n) {
        int l_weight = heap_elements[l].weight;

        if (smallest_weight < l_weight) {
            smallest = l;
        } else if (smallest_weight == l_weight) {
            if (heap_elements[smallest].key < heap_elements[l].key) {
                smallest = l;
            }
        }
    }

    if (r < n) {
        int r_weight = heap_elements[r].weight;

        if (smallest_weight < r_weight) {
            smallest = r;
        } else if (smallest_weight == r_weight) {
            if (heap_elements[smallest].key < heap_elements[r].key) {
                smallest = r;
            }
        }
    }

    if (i != smallest) {
        swap(&heap_elements[i], &heap_elements[smallest]);
        heapify(heap_elements, n, smallest);
    }
}
\end{minted}
\caption{Método de ordenação (construção do heap)}
\end{listing}

Ressalta-se, contudo, que este é uma das partes do método de ordenação pois
ela é composta de outra parte tão igualmente importante a supracitada pois
de outra forma não seria realizada ordenação nenhuma. Portanto, com essas
operações elementares inicia-se o desenvolvimento do algoritmo para ordenação
externa que se utiliza do algoritmo descrito na seção anterior para executar
o método de ordenação externa de substituição por meio de seleção.

\part{Referências}
\label{sec-4}

\noindent
CORMEN, Thomas et al. \textbf{Algoritmos}. 3. ed. Elsevier, 2012.

\noindent
ZIVIANI, Nivio. \textbf{Projeto de algoritmos com implementações Pascal e C}. 4. ed.
Sâo Paulo: Pioneira, 1999.
% Emacs 25.3.1 (Org mode 8.2.10)
\end{document}
